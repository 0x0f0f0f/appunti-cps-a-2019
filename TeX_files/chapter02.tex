\chapter{Spazio Probabilizzato}

Uno spazio probabilizzato è definito come

\[ (\Omega, F, \mathcal{P}) \]

$ \Omega $ è l'insieme degli eventi elementari, ad esempio in un lancio di un dado $ \Omega = \{1,2,3,4,5,6\} $

$ \mathcal{P}(A)$, ovvero le parti di A, sono tutti gli insiemi che posso costruire a partire dagli elementi di A. Ad esempio:
\begin{align*}
& A = \{0, 1\} \\
& \mathcal{P}(A) = \{0, \{0, 1\}, \{0\}, \{1\}\}
\end{align*}

$ F \subseteq \mathcal{P}(\Omega) $ è un sottoinsieme delle parti di omega, chiuso rispetto a intersezione, unione e complementare.
\[ A,B \in F \implies \begin{cases}
A \cup B \\
A \cap B \\
A^C, B^C
\end{cases} \in F\]

Se $ A_1, \dots, A_n \subset F \implies \bigcup\limits_{n \in \N} A_n \in F $

Se $ F $ comprende queste proprietà si dice che è una $ \sigma $ Algebra (tribù)

La probabilità $ P $ è una funzione definita come

\begin{align*}
&P : F \to [0,1] \\
& P(\Omega) = 1 \\
& \forall i \neq j . A_i \cap A_j \neq 0 \implies P \left( \bigcup\limits_{n} A_n \right) = \sum_{n} P(A_n)
\end{align*}

Dato $ \Omega $ insieme finito, allora $ F = \mathcal{P}(\Omega) $ allora la probabilità sarà
\[  P (A \subset F) = \dfrac{\#A}{\#\Omega} \]

\paragraph{Esempio: Lotteria di De' Finetti}
Si assiste all'estrazione di un numero $ n \in \N $ casuale. Supponiamo che ogni numero abbia la stessa probabilità di essere estratto.

Dato un altro naturale $ m \in \N $

\begin{align*}
P_n = P(n) = \text{ La probabilità di estrarre il numero n}
\end{align*}

Vogliamo che 

\begin{align*}
& 0 \leq P_n \leq 1 \\
& P_m = P_n \forall n = m
\end{align*}

Quindi $ 1 = P(\Omega) = P \left( \bigcup\limits_{n \in \N} n \right) = \sum_{n \in \N} P_n $

Assumendo $ P_n = 0, \forall n $ allora $ \sum_{n} p_n = 0 $

Se $ P_n = c > 0, \forall n $ allora $ \sum_{n} p_n = \sum_{n} c = + \infty $

Ciò significa che nella lotteria di De' Filetti è impossibile che ogni numero sia equiprobabile.

\paragraph{Dimostrazione}

La terna $ (\Omega, F, \mathcal{P}) $ si può dimostrare.

\[ P(A^C) = 1 - P(A) \impliedby \begin{cases}
A \cap A^C \neq 0 \\
A \cup A^C = \Omega \\
P(\Omega) = 1
\end{cases} \]

\[ P(A \cup B) = P(A) + P(B) - P(A \cap B)\]

\[ A \subseteq B \implies P(A) \leq P(B) \]

\paragraph{Densità di Probabilità}

Definiamo $ \{p_n\}_{n \in \N} $ con $ p_n \in \R $, detta densità di probabilità come $ p_n \geq 0 $ e $ \sum_{n \in \N} p_n = 1 $ 

Se $ \{p_n\} $ è una densità di probabilità discreta $ \implies \left( \N, \mathcal{P}(\N), P\right) $ è uno spazio probabilizzato. Vale anche per eventi non equiprobabili.

\paragraph{Definizione, indipendenza degli eventi} 

$ P(A|B) \equiv \dfrac{P(A \cap B)}{P(B)} $ 

Allora $ A,B $ indipendenti se $ P(A|B) = P(A) $

È vero quindi che $ P(A|B) = P(A) \implies P(B|A) = P(B) $? Sì se $ A,B $ sono indipendenti.

\[ P(A|B) = \dfrac{P(A \cap B)}{P(B)} = P(A) \]
\[ P(B|A) = \dfrac{P(B \cap A)}{P(A)} = P(B) \]
\[ P(A) = \dfrac{P(A \cap B)}{P(B)} \]
\[ P(B|A) = \dfrac{P(B \cap A)}{P(A)} = \dfrac{P(B) \cdot P(A)}{P(A)} = P(B)\]
\[ P(A \cap B)  = P(A) \cdot P(B) = P(B \cap A) \]

Se $ A_1, \dots, A_n $ sono indipendenti:
\[ \forall i_1, \dots, i_k \land k \leq n \] (Con indici tutti diversi) allora vale \[ P(A)\cap \left(A_i1,\dots,A_ik\right) = P(A_i1) \cdot \dots \cdot P(A_ik) \] 

\paragraph{Esercizio teorico}

$ A, B $ indipendenti $ \implies \begin{cases}
A, B^C \\
A^C, B^C \\
A^C, B
\end{cases} $ indipendenti

\textbf{Dimostrazione:}
\[ A = (A \cap B) \cup (A \cap B^C)  \]
\[ P(A) = P(A \cap B) + P(A \cap B^C) = P(A) \cdot P(B) + P(A \cap B^C) \]
\[ P(A \cap B^C) = P(A) - P(A) \cdot P(B) = P(A)\left[1 - P(B)\right] = P(A) \cdot P(B^C) \]

$ A $ e $ B^C $ sono indipendenti \enddim