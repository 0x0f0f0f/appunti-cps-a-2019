\chapter{Spazio Probabilizzato}

\section{Lo Spazio Probabilizzato}
Uno spazio probabilizzato è un costrutto matematico che modella un processo del mondo reale o "esperimento", consistente in degli stati che occorrono casualmente. Viene costruito su una situazione o esperimento particolare, Uno spazio probabilizzato è definito come una terna:

\[ (\Omega, F, \mathcal{P}) \]

$ \Omega $ è l'insieme degli eventi elementari, ovvero tutti i risultati possibili, ad esempio in un lancio di un dado $ \Omega = \{1,2,3,4,5,6\} $

$ \parts{A}$, ovvero le parti di A, sono tutti gli insiemi che posso costruire a partire dagli elementi di A. Ad esempio:
\begin{align*}
& A = \{0, 1\} \\
& \parts{A} = \{0, \{0, 1\}, \{0\}, \{1\}\}
\end{align*}

$ F \subseteq \parts{\Omega} $ è un sottoinsieme delle parti di omega, chiuso rispetto a intersezione, unione e complementare.
\[ A,B \in F \implies \begin{cases}
A \cup B \\
A \cap B \\
A^C, B^C
\end{cases} \in F\]

Se $ A_1, \dots, A_n \subset F \implies \bigcup\limits_{n \in \N} A_n \in F $

Se $ F $ comprende queste proprietà si dice che è una $ \sigma $ Algebra (tribù)

La probabilità $ P $ è una funzione definita come

\begin{align*}
&P : F \to [0,1] \\
& P(\Omega) = 1 \\
& \forall i \neq j . A_i \cap A_j \neq \emptyset \implies P \left( \bigcup\limits_{n} A_n \right) = \sum_{n} P(A_n)
\end{align*}

Dato $ \Omega $ insieme finito, allora $ F = \parts{\Omega} $ allora la probabilità sarà
\[  P (A \subset F) = \dfrac{\#A}{\#\Omega} \]

\paragraph{Esempio: Lotteria di De' Finetti}
Si assiste all'estrazione di un numero $ n \in \N $ casuale. Supponiamo che ogni numero abbia la stessa probabilità di essere estratto.

Dato un altro naturale $ m \in \N $

\begin{align*}
P_n = P(n) = \text{ La probabilità di estrarre il numero n}
\end{align*}

Vogliamo che 

\begin{align*}
& 0 \leq P_n \leq 1 \\
& P_m = P_n \forall n = m
\end{align*}

Quindi $ 1 = P(\Omega) = P \left( \bigcup\limits_{n \in \N} n \right) = \sum_{n \in \N} P_n $

Assumendo $ P_n = 0, \forall n $ allora $ \sum_{n} p_n = 0 $

Se $ P_n = c > 0, \forall n $ allora $ \sum_{n} p_n = \sum_{n} c = + \infty $

Ciò significa che nella lotteria di De' Filetti è impossibile che ogni numero sia equiprobabile.

\paragraph{Dimostrazione}

La terna $ (\Omega, F, \mathcal{P}) $ si può dimostrare.

\[ P(A^C) = 1 - P(A) \impliedby \begin{cases}
A \cap A^C \neq \emptyset \\
A \cup A^C = \Omega \\
P(\Omega) = 1
\end{cases} \]

\[ P(A \cup B) = P(A) + P(B) - P(A \cap B)\]

\[ A \subseteq B \implies P(A) \leq P(B) \]

\paragraph{Densità di Probabilità }

Definiamo $ \{p_n\}_{n \in \N} $ con $ p_n \in \R $ come funzione, detta densità di probabilità come $ p_n \geq 0 $ e $ \sum_{n \in \N} p_n = 1 $ 

Se $ \{p_n\} $ è una densità di probabilità discreta $ \implies \left( \N, \parts{\N}, P\right) $ è uno spazio probabilizzato. Vale anche per eventi non equiprobabili.

\paragraph{Definizione, indipendenza degli eventi} 

$ P(A|B) \equiv \dfrac{P(A \cap B)}{P(B)} $ 

Allora $ A,B $ indipendenti se $ P(A|B) = P(A) $

È vero quindi che $ P(A|B) = P(A) \implies P(B|A) = P(B) $? Sì se $ A,B $ sono indipendenti.

\[ P(A|B) = \dfrac{P(A \cap B)}{P(B)} = P(A) \]
\[ P(B|A) = \dfrac{P(B \cap A)}{P(A)} = P(B) \]
\[ P(A) = \dfrac{P(A \cap B)}{P(B)} \]
\[ P(B|A) = \dfrac{P(B \cap A)}{P(A)} = \dfrac{P(B) \cdot P(A)}{P(A)} = P(B)\]
\[ P(A \cap B)  = P(A) \cdot P(B) = P(B \cap A) \]

Se $ A_1, \dots, A_n $ sono indipendenti:
\[ \forall i_1, \dots, i_k \land k \leq n \] (Con indici tutti diversi) allora vale \[ P(A)\cap \left(A_i1,\dots,A_ik\right) = P(A_i1) \cdot \dots \cdot P(A_ik) \] 

\paragraph{Esercizio teorico}

$ A, B $ indipendenti $ \implies \begin{cases}
A, B^C \\
A^C, B^C \\
A^C, B
\end{cases} $ indipendenti

\textbf{Dimostrazione:}
\[ A = (A \cap B) \cup (A \cap B^C)  \]
\[ P(A) = P(A \cap B) + P(A \cap B^C) = P(A) \cdot P(B) + P(A \cap B^C) \]
\[ P(A \cap B^C) = P(A) - P(A) \cdot P(B) = P(A)\left[1 - P(B)\right] = P(A) \cdot P(B^C) \]

$ A $ e $ B^C $ sono indipendenti \enddim

\paragraph{Esercizio: Lancio di due dadi}
Lancio due dadi, uno rosso ed uno nero.
$ \Omega = (r, n) $ dove $ r = 1,2,3,4,5,6 $ e $ n = 1,2,3,4,5,6 $ 
Definiamo lo spazio probabilizzato con $ F = \parts{\Omega}, \omega \in \Omega $. Ad esempio $ \prob{n} = \dfrac{1}{36} $

Probabilità che il rosso sia 3 sapendo che rosso + nero fa 6
\[ \prob{r=3 | r+n = 6} = \dfrac{\prob{r=3 \cap r + n = 6}}{\prob{r+n=6}} = \dfrac{\dfrac{1}{36}}{\dfrac{5}{36}} = \dfrac{1}{5}\] 

Probabilità che il rosso sia pari sapendo che rosso + nero fa 6

\[ \prob{r = \text{ pari} | r+n = 6} = \dfrac{\prob{r \text{ pari} \cap r+n = 6}}{\prob{r+n = 6}} = \dfrac{\dfrac{2}{36}}{\dfrac{5}{36}} \]

\paragraph{Esercizio: Gioco di Monty Hall}
Riprendendo il gioco delle tre porte definiamo lo spazio probabilizzato: Formalizzo di aver scelto la porta 3. $ \Omega = (x,y) $ dove $ x = 1,2,3 $ è la porta vincente e $ y = 1,2 $ è la porta perdente che è stata aperta dal presentatore.

Gli eventi impossibili saranno $ \prob{1,1} = 0, \prob{2,2} = 0 $

% TODO tabella

Dalla tabella otteniamo che
\[ \prob{x=1} = \prob{x=2} = \prob{x=3} = \dfrac{1}{3} \]
\[ \prob{x=1,y=2} = \dfrac{1}{3} \]
\[ \prob{x=2,y=1} = \dfrac{1}{3} \]
\[ \prob{x=3,y=1} = \prob{x=3,y=2} = \dfrac{1}{6} \]

Quindi $ \prob{y=1} = \prob{y=2} = \dfrac{1}{2} $

Se scelgo la porta 3, suppongo venga aperta la 2. Se non cambio e vinco $ (x = 3) $ allora

\[ \prob{x=3 | y=2} = \dfrac{\prob{(3,2)}}{\prob{y=2}} = \dfrac{\dfrac{1}{6}}{\dfrac{1}{2}} = \dfrac{1}{3} \]

Se scelgo la porta 3, suppongo venga aperta la 1 e cambio allora:

\[ \prob{x=1 | y=2} = \dfrac{\prob{(1,2)}}{\prob{y=2}} = \dfrac{\dfrac{1}{3}}{\dfrac{1}{2}} = \dfrac{2}{3} \]

\subsection{Formula di fattorizzazione}
Supponiamo di avere una famiglia di insiemi $ B_1, \dots, B_n $ con $ n \in \N $ che è detta una partizione finita di $ \Omega $ (insieme fondamentale). Voglio che $ \forall i . B_i \in F $ e che $ \forall i \forall j\neq i . B_i \cap B_j = \emptyset $ e anche che $ \bigcup_{i=1}^{n} B_i = \Omega $

\paragraph{Teorema}

Sia $ \{B_i\}_{i=1, \dots, n} $ parte finita di $ \Omega $ e sia $ \prob{B_i} > 0 $

\[ \implies  \prob{A} = \sum_{i=1}^{n} \prob{A|B_i} \cdot \prob{B_i} \]

\paragraph{Dimostrazione} 

\[ \prob{A} =  \sum_{i=1}^{n} \prob{A \cap B_i} = \sum_{i=1}^{n} \prob{A|B_i} \cdot \prob{B_i} \]

\paragraph{Condizionamento ripetuto}
Dati $ A_1, \dots, A_n $ eventi, allora 
\[ \prob{A_1 \cap \dots \cap A_n} = \prob{A_1} \cdot \prob{A_2|A_1} \cdot \prob{A_3| A_1 \cap A_2} \cdot \prob{A_n | A_1 \cap \dots \cap A_{n-1}}\]

\subsection{Teorema di Bayes}

% TODO approfondisci da wiki

Dati due eventi $ A,B $ con probabilità non nulla $ \prob{A} > 0 $ e $ \prob{B} > 0 $ allora 

\[ \prob{B|A} = \dfrac{\p{A|B} \cdot \p{B} }{\p{A}}\]

% TODO dimostrazione

\section{Esercizi}

\paragraph{Esercizio: Vino}

Un produttore di vino produce due vini (bianco $ B $ e rosso $ R $) e vende in Francia ($ F $) e Germania ($ G $). 
Le vendite sono $ 1/3 $ per la Francia e $ 1/3 $ per la Germania. $ 3/4 $ delle richieste dalla Francia sono vino bianco. $ 1/4 $ delle richieste dalla Francia sono vino rosso. $ 1/2 $ delle richieste dalla Germania sono vino bianco. $ 1/2 $ delle richieste dalla Germania sono vino rosso.

Utilizzando la formula di partizione troviamo la probabilità che una richiesta dalla Germania sia vino bianco. $ \p{\text{richiesta dalla germania} = B} $

\[ \p{B} = \p{B|G} \cdot \p{G} + \p{B|F} \cdot \p{F} = \dfrac{1}{3} + \dfrac{1}{4} = \dfrac{7}{12} \]

\paragraph{Esercizio, esami e studenti}

Abbiamo 3 livelli di preparazione: Ottimo, Buono e Scarso.
Un esito dell'esame è Promosso o Respinto.

\[ \p{\text{Promosso} | \text{Ottimo}} = 0,995\]
\[ \p{\text{Promosso} | \text{Scarso}} = 0,3\]
\[ \p{\text{Promosso} | \text{Buono}} = 0,8\]

Uno studente prova l'esame e viene respinto. Qual'è la probabilità che aveva di avere una preparazione scarsa? $ \p{\text{Scarso}|\text{Respinto}} $

Prima calcoliamo la probabilità di essere respinti.

\[ \p{R} = \p{R|O} \cdot \p{O} + \p{R | B} \cdot \p{B} + \p{R|S} \cdot \p{S} = 0,302 \]

Senza informazioni aggiuntive $ \p{O} = \p{B} = \p{S} = \dfrac{1}{3} $

La probabilità di essere respinto è $ \p{R} = 0,302 $ quindi 

\[ \p{S | R} = \dfrac{\p{R|S}\cdot \p{S}}{\p{R}} = \dfrac{0,7 \cdot 1/3}{0,302} = 0,773 \]

\paragraph{Esercizio alternativo}

% TODO per casa

Qual'è la probabilità che lo studente aveva di avere una preparazione scarsa, sapendo che è stato respinto e sapendo che le probabilità dei voti sono: $ \p{O} = \dfrac{1}{6}, \p{B} = \dfrac{2}{3}, \p{S} = \dfrac{1}{6} $

\paragraph{Esercizio: Malattia con fattore di rischio}
La probabilità di ammalarsi di un soggetto a rischio $ (R) $ è $ 0,2 $, mentre la probabilità di ammalarsi di un soggetto non a rischio $ (N) $ è $ 0,006 $. Il $ 15\% $ della popolazione sono soggetti a rischio. Un malato si denota con $ M $ mentre uno sano con $ S $.

Vogliamo sapere 

\begin{enumerate}
	\item $ \p{\text{Soggetto casuale sia malato}} = $
	
	$ \p{M} = \p{M|R} \cdot \p{R} + \p{R|N} \cdot \p{N} = 0,35 $ 
	
	$ \p{M} = 0,2 \cdot 0,15 + 0,006 \cdot 0,85 = 0,35 $
	
	\item $ \p{\text{Soggetto malato fosse a rischio}} = $
	
	$ \p{R|M} = \dfrac{\p{M|R} \cdot \p{R}}{\p{M}} = \dfrac{0,2 \cdot 0,15}{0,35} = 0,855 $
	
	\item $ \p{\text{Soggetto soggetto sano sia a rischio}} = $
	
	$ \p{R|S} = \dfrac{\p{S|R} \cdot \p{R}}{\p{S}} = \dfrac{(1-0,2) \cdot 0,15}{(1-0,35)} = 0,124$
\end{enumerate}

\paragraph{Osservazione}

La probabilità che l'evento $ A^c $ ($ A $ complementare) si verifichi sapendo B è $ \p{A^c|B} = 1 - \p{A|B} $ 
\textbf{mentre invece} la probabilità di $ A $ sapendo $ B^c $ è $ \p{A|B^c} \neq 1 - \p{A|B} $

\paragraph{Esercizio: Test}
Prendendo $ S = $ soggetti sani, $ M = $ soggetti malati, $ T^- = $ test positivo, $ T^+ = $ test positivo.

La specificità di un test è $ \p{T^-|S} $. Una specificità alta implica pochi falsi positivi. La sensibilità è $ \p{T^+|M} $. Una sensibilità alta implica pochi falsi negativi.
